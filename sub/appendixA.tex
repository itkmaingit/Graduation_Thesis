\chapter{付録}
\section{ある盤面を与えたとき全てのグラフを要素に持つ集合$A$がいつでも一意に取れることの証明}\label{section:GraphUnique}
ある盤面を与えたとき, $A$(\cref{equation:A})がいつでも一意であることを証明する.
\begin{proof}
  $A=\{G_1,G_2,...,G_a\}=\{G_z\}$だが, $C=\{L_1,L_2,...,L_a\}=\{L_z\}$が一意であることを示すことと同値である. 盤面$B$を与えたとき$C$, $C'$が定まるとする. $C\ni  L_z$を取ったとき, $A'\ni  L'_{z'} \mbox{ s.t. $G_z \cap L'_{z'} \neq \emptyset $}$なる$L'_{z'}$が存在する. $L_z$, $L'_{z'}$が線であるから$L_z\cap L'_{z'}$も線である. いま, $L_z\subset L_z\cap L'_{z'}$であるが$L_z$が最大である仮定から$L_z\supset L_z\cap L'_{z'}$である. よって, $L'_{z'}$も同様の議論より, $L_z= L_z\cap L'_{z'}=L'_{z'}$である. これが$C$に含まれる線について全て成立することより, $C=C'$. $C=C'$から$A=A'$である. 以上の議論より, 題意は示された.
\end{proof}

\section{完成盤面から完成可能盤面を生成する具体的手法}\label{section:GenericAlgorithm}
藤原 \cite{Fujiwara2022}が述べた, HIを用いることにより完成盤面から完成可能盤面を生成する手法について以下で説明する. ただし, 例として藤原 \cite{Fujiwara2022}と同様に数独(ナンバープレイス)を用いる. 数独(ナンバープレイス)のパズルルールは以下のとおりである. \cite{web:Sudoku}

\begin{enumerate}
  \item あいているマスに, 1から9までの数字のどれかを入れます.
  \item タテ列 (9列あります), ヨコ列 (9列あります), 太線で囲まれた3$\times$3のブロック(それぞれ9マスあるブロックが9つあります)のどれにも1から9までの数字が1つずつ入ります.
\end{enumerate}

\begin{example}[遺伝的アルゴリズム]
  数独(ナンバープレイス)の完成盤面から完成可能盤面を生成する遺伝的アルゴリズムは以下のものである.
  \begin{enumerate}
    \item タテ列, ヨコ列, 太線で囲まれた3$\times$3のブロックの内部に数字の被りがない完成盤面$X$を1つ用意する.
    \item その盤面から, HIの具体的な位置を決め, その細胞の状態を全て$null$にする.
    \item 2. の盤面は1. の完成盤面を解として含むが, その盤面は完成可能盤面と限らない.
    \item ソルバーがその盤面を解き,
          \begin{enumerate}
            \item 唯一解であれば4. は完成可能盤面であり, 数独(ナンバープレイス)の問題である.
            \item その盤面に対応する完成盤面が存在しない, あるいは前の子孫より完成盤面が増えたらその盤面を捨てて5. へ進む(初回ならその盤面を5. へと進める. ).
            \item その盤面に対応する完成盤面が前の子孫より減ったらその盤面を採用して5. へ進む(初回ならその盤面を5. へと進める. ).
          \end{enumerate}
    \item HIの具体的な位置を変えることにより, 別の完成盤面の存在を仮定する. そのような操作を施した別の盤面を用意して4. へ戻る.
  \end{enumerate}
\end{example}

藤原 \cite{Fujiwara2022}はこの遺伝的アルゴリズムを用いて多くのペンシルパズルの問題を生成できると述べている.
