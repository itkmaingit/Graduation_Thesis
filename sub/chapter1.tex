\chapter{{\rm \bf Introduction}}
\section{研究背景, 目的}
本研究は, ペンシルパズルのパズルルール(以下パズルルール)の自動作成に対する課題を解決するためのものである. 本論文ではHertingら\cite{Herting2004}が行ったようにパズルルールを数学的な定義を行い, その定義を用いて新規のパズルルールを提案した. 研究の調査段階においてはパズルルールを0から生成した例が確認されていない.本研究に近しいものとしてBrowne\cite{Browne2010}は二人有限確定完全情報ゲームを既存のゲームから組み合わせて生成した. しかし, この研究は0からゲームルールを生成したものではなく, また工学的手法で生成しているという点で本研究とは異なる. また, 多くの研究(吉仲ら\cite{Yoshinaka2012}やMantereら\cite{Mantere2007})にてあるパズルルールが存在したとき, そのパズルルールに則った問題を作成した先行研究が多数確認されている. しかし, こちらはパズルルールの作成ではなく, パズルルールに則った具体的な問題を生成するものであり, 研究としては異なるものである. Herting	ら\cite{Herting2004}の研究はスリザーリンクというパズルにおいてのみパズルの盤面と, それが満たすべき条件を数学的に定義し, 問題を効率的に解くためのrule-based approachを行ったものである. これも上記の例と同様に, 本研究とは主旨が異なるものである.

本研究は, 先行研究では行われていないパズルルールの自動作成に対する課題を解決するための研究である. この研究により, 関連する分野における問題解決における価値を提供することができる. 具体的には, 商業用ゲームへの移植や, ネットワークセキュリティへの応用などが考えられる. Barbara\cite{Barbara2020}は「プレイヤーはゲームをプレイする際に, 同じゲーム性の繰り返しからいわゆるマンネリ化を感じる傾向にあるところを, パズルを導入すれば緩和できる」と述べている. さらに, Liangら\cite{Liang2013}はネットワークセキュリティに対する問題をゲーム理論的アプローチで考察している. パズルはゲーム理論と密接な関係にあり, 	本研究はこれら分野への応用が期待される.

\section{定義}\label{section:IntroDefinition}
ペンシルパズルとは「サイズ$m\times n$の平面グリッド上に存在する概念や状態に対し, 盤面が満たされるべき条件を提示した問題それらに付随する概念」のことである. パズルルールとは, その問題が存在するための必要十分条件とする. 本研究ではまず, グリッド上の格子点, 細胞, 縦辺, 横辺という四つの概念を位置という用語を用いて数学的に定義した(\cref{figure:Board}). そして, それぞれの位置には未知と既知という状態が存在し, それらには具体的な値として与えられることについて言及を行った. その値が含まれる集合を
\begin{definition}[\textit{codomain}]
	位置に対応する状態がとりうる集合のことを\textbf{\textit{codomain}}と定義する.
\end{definition}
とした. それらを定義したとき, 盤面が$B\coloneqq\Bigl\{\,\{\,p_{i,j}\,\},\{\,c_{i,j}\,\},\{\,e_{i_y,j_y,y}\,\}\,\Bigr\}$と数学的な記述が可能になることを示した.  ただし, $\Bigl\{\,\{\,p_{i,j}\,\},\{\,c_{i,j}\,\},\{\,e_{i_y,j_y,y}\,\}\,\Bigr\}$は各位置の状態の列である. そして, パズルルールにおいて盤面が満たすべき条件を
\begin{definition}[完成盤面と\textit{conditions}]
	全ての位置の状態が既知である盤面を与えたとき, 状態が満たしているべき条件を\textbf{\textit{conditions}}と定義する.
\end{definition}
として定義することで, パズルの問題の完成盤面がこの組み合わせで全て記述できることを示した. ただし, 本文では問題と完成盤面という用語をさらに厳密に定義する. さらに, パズルルールには問題が存在するので, 完成盤面から問題を作成するために以下二つの定義を導入した.
\begin{definition}[\textit{Hidden Information}(HI)]
	ある完成盤面があったとき, ソルバーから隠す位置の部分集合を\textbf{\textit{Hidden Information}(HI)}と定義する.
\end{definition}
\begin{definition}[\textit{identification}]
	未知の状態と, 既知の状態の具体的な値がソルバーに同一の状態として伝えられるその対応を\textbf{\textit{identification}}として定義する.
\end{definition}

\section{結果}
\cref{section:IntroDefinition}で定義した四つの定義を用いて, パズルルールがそれら\textit{codomain}, \textit{conditions}, HI, \textit{identification}の組み合わせで定義されることを主張した(\cref{definition:PuzzleRule}).そして, その妥当性を既存のパズルルールにおいても成り立っていることを示し, 新規のパズルルールを作成できることで検証した.
