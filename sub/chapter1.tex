\chapter{緒言}
\section{研究背景}
一般にパズルとは,
\begin{quote}
  puzzle, a problem that may take many forms, including games and toys, and is solved through knowledge, ingenuity, or other skills.
\end{quote}
とRaikar\cite{Raikar2022}に述べられているように,解答者が各々の知識や技術を用いて解く問題のことである。各ジャンルの中で様々な形式を取るのでそれ以上のことを一概に定義することはできないが、共通することとしてある制約の中で各問題に用意された唯一解を導くことということである.つまりパズルは唯一解を持つことが必要条件であり,唯一解でない問題に関しては「ジレンマ」や「パラドックス」と論理学的に呼ばれている\cite{wiki:puzzle}。パズルの歴史は深く,人々が解く問題に対し,具体的にパズルと名称がついたのは16世紀の終わり頃であり,それから人々の間に娯楽として浸透するようになり、エンターテイメントとして現代社会においてかなり重要なものの一つである。

本研究ではその中でもロジックパズル、ペンシルパズルと呼ばれるものを対象とするが、ペンシルパズルに関する研究はこれまでにかなり進んでいる。特に特定のパズルルールに対し効率良く解くアルゴリズムやそれの計算量やNP困難性、あるいは問題を生成する研究は終了しており、現在は特定のパズルルールの問題の難易度評価などが研究対象とされている。

しかしながら研究の調査段階においてはメタ的にペンシルパズルのパズルルールを自動生成した例が確認されていない.本研究に近しいものとして二人有限確定完全情報ゲームを既存のゲームから組み合わせて生成した研究やGame Description Language(GDL)を用いて任意のパズルルールの生成を行うことを可能にした研究が存在するが前者は0からゲームルールを生成したものではない点で、後者はパズルルールを状態とアクションの2つの相互作用で定義できるとしてそれを論理型言語で記述するようにしたものであり、パズルルールの詳細に触れているわけではない点で本研究と異なる。


\section{研究目的}
本研究の目的はペンシルパズルのパズルルールを生成する手法を開発することである。そのためにまず、ペンシルパズルにおけるパズルルールを「完成盤面における盤面上の変数が取りうる解の集合(codomain)」、「完成盤面が満たしているべき必要十分条件(conditions)」、「ソルバーに隠す盤面上の変数の解(Hide Information)」と3つの要素に分割し
