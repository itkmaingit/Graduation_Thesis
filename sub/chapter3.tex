\chapter{仮定, 手法}
\label{chapter:3}

\section{仮定と手法}

\cref{chapter:Prepare}で定義したパズルルールによって既存のパズルルールとが説明できることを以下の例にて説明する.
実証例として四角に切れとスリザーリンクを用いる. 四角に切れのパズルルールは以下のものである\cite{web:Sikakunikire}

\begin{enumerate}
  \item 盤面を長方形(正方形)に分割します.
  \item 数字は, 1マスの面積を1とした, 長方形の面積です. 4と書いてあるマスを含む長方形は,  1$\times$4, 2$\times$2, 4$\times$1のどれかになります.
  \item 切るのは線の上で, どの長方形にも数字が1つずつ入ります.
\end{enumerate}

\begin{example}

\end{example}
\begin{example}

つまり上記より,完成盤面が1つ以上存在すれば未完成盤面が生成でき,問題とすることができる.つまり,節(\ref{subsection:PuzzleDefinition})で定義した3つの要素($\mathbf{codomain,conditions,HI}$)を具体的に構成し,そのパズルルールに対し実際に完成盤面を1つ以上存在することを示すことがパズルルールの生成手法と言える.


\section{新しいパズルルールの作成}
