\chapter{仮定,手法}
\label{chapter:3}

\section{仮定と手法}
\subsection{パズルルールの定義}\label{subsection:PuzzleDefinition}
前章\cref{chapter:Prepare}で定義した用語を用いることによってペンシルパズルにおけるパズルルールという用語の定義を行う.
改めてペンシルパズルにおけるパズルルールとは完成盤面において解が満たしているべき必要十分条件($\mathbf{conditions}$)と,その解が取りうる値の範囲($\mathbf{codomain}$),そしてソルバーに向けて隠される情報($\mathbf{HI}$)の組み合わせであると定義する.
これが既存のパズルルールと直感的に違わないことから,次節でパズルルールの自動生成の手法を提示する.
\subsection{パズルルールの生成の具体的手法}
ペンシルパズルのような数理的パズルにおいて,進化計算,特に遺伝的アルゴリズムにより問題が自動生成できると藤原\cite{Fujiwara2022}にある.ここで,ペンシルパズルにおける具体的なアルゴリズムを以下に記す.ただし,用語は本研究のものを用いたものとしている.例として\cite{Fujiwara2022}と同様に数独(ナンバープレイス)を用いる.ただし,数独(ナンバープレイス)のパズルルールは以下のとおりである.\cite{web:Sudoku}

\begin{enumerate}
  \item あいているマスに, 1から9までの数字のどれかを入れます.
  \item タテ列 (9列あります), ヨコ列 (9列あります), 太線で囲まれた3$\times$3のブロック (それぞれ9マスあるブロックが9つあります) のどれにも1から9までの数字が1つずつ入ります.
\end{enumerate}

\begin{example}\textup{遺伝的アルゴリズムの具体例}
  \begin{enumerate}
    \item 縦,列,$3\times3$の部屋に数字の被りがない完成盤面$X$を1つ用意する.
    \item その盤面から,$\mathbf{HI}$の具体的な列を決め,その部分を全て$null$にする.
    \item 2.の盤面は1.の完成盤面を解として含むが,一般にその完成盤面は唯一解ではないので未完成盤面ではない.
    \item ソルバーがその盤面を(論理的に)解き,
          \begin{enumerate}
            \item 空白マスが0(全て未知情報を既知情報に論理的に変換させられる)なら,それは唯一解なので4.は未完成盤面であり,パズルの問題である.
            \item 空白マスが増えたり,矛盾が生まれたら(既知情報に出来ない$\{c_{i,j}\}$が前のステップより増えたら)その盤面を捨てて5.に進む.(初回ならその盤面を5.へ)
            \item 空白マスが減ったら,(既知情報に出来ない$\{c_{i,j}\}$が前のステップより減ったら)その盤面を採用して5.へ進む.(初回ならその盤面を5.へ)
          \end{enumerate}
    \item $\mathbf{HI}$の具体的な列に別の列を採用するなどして,別の完成盤面の存在を仮定した別の盤面を用意して4.へ戻る.
  \end{enumerate}
\end{example}

つまり上記より,完成盤面が1つ以上存在すれば未完成盤面が生成でき,問題とすることができる.つまり,節(\ref{subsection:PuzzleDefinition})で定義した3つの要素($\mathbf{codomain,conditions,HI}$)を具体的に構成し,そのパズルルールに対し実際に完成盤面を1つ以上存在することを示すことがパズルルールの生成手法と言える.


\section{新しいパズルルールの作成}
