\chapter{手法}\label{chapter:3}
\section{章の概略}
今章では, \cref{chapter:Prepare}で定義したパズルルールがよく定義できているか確認を行う. \cref{section:ExistsPuzzleRule}では既存のパズルルールのうち\cref{chapter:Prepare}で定義したパズルルールとして説明できるものがあることを実証する.  さらに\cref{section:NewPuzzleRule}ではパズルルールの自動作成の足掛けとして新しいパズルルールが作成できることを説明する.
そして\cref{section:CannotDescribePuzzleRule}では\cref{chapter:Prepare}で定義したパズルルールで既存のパズルルール全てを記述できないことを説明する.

\section{既存のパズルルール}\label{section:ExistsPuzzleRule}
今節では既存のパズルルールが\cref{chapter:Prepare}で定義したパズルルールで記述できることを実証する. 実証方法として, ある既存のパズルルールの完成盤面全体が\textit{codomain}と\textit{conditions}で全て余すことなく記述でき, そのパズルルールとして認められない完成盤面がないことを証明する. そして\textit{HI}により未完成盤面が表現できることを証明することにより実証する.
実証例としてナンバーリンクとスリザーリンクを用いる. ナンバーリンクのパズルルールは以下のものである\cite{web:NumberLink}.

\begin{enumerate}
  \item 白マスに線を引いて, 同じ数字どうしをつなげましょう.
  \item 線は, マスの中央を通るようにタテヨコに引きます. 線を交差させたり, 枝分かれさせたりしてはいけません.
  \item 数字の入っているマスを通過するように線を引いてはいけません.
  \item 1マスに2本以上の線を引いてはいけません.
\end{enumerate}

\begin{example}[スリザーリンクの数学的記述]
  スリザーリンクの\textit{codomain}, \textit{conditions}, \textit{HI}にはそれぞれ\cref{example:SlitherLinkCodomain}, \cref{example:SlitherLinkConditions}, \cref{example:SlitherLinkHiddenInformation}を用いる.
\end{example}

\begin{example}[ナンバーリンクの数学的記述]
  ナンバーリンクは, 以下で示す\textit{codomain}, \textit{conditions}で定めた盤面をシフトすることによりナンバーリンクの完成盤面を余すことなく記述することができる.
  \begin{align}
     & \mathbb{P}=\emptyset                                         \\
     & \mathbb{C}=\{\textit{null}, \emptyset ,x_1,x_2,\ldots, x_a\} \\
     & \mathbb{E}=\{\textit{null},0,1\}                             \\
  \end{align}

  \begin{align}
     & B\ni \forall p(i,j),1\le \textit{cross}(p(i,j))\le 2                          \\
     & B\ni \forall p(i,j),  \textit{cross}(p(i,j))= 2 \Rightarrow p_{i,j}=\emptyset \\
     & \forall G_z\ni P_z,        |\{p(i,j)\mid cross(p(i,j))=1\}|=2                 \\
     & \forall G_z\ni P_z,     \textit{cross}(p(i,j))= 1 \Rightarrow p_{i,j}=x_a     \\
  \end{align}
  \begin{align}
     & \{e_{i,j}\}                      \\
     & e:\textit{null}\leftrightarrow 0
  \end{align}

\end{example}

\section{新しいパズルルールの作成}\label{section:NewPuzzleRule}
今節では\cref{chapter:Prepare}で定義したパズルルールで, 新しいパズルが作成できることを説明する. ただし, \cref{theorem:Equivalent}より, 完成盤面を一つ提示することが問題の提示と同値となることに注意する. まず, 以下のように\textit{codomain}, \textit{codomain}, \textit{codomain}を次のように選ぶ.

\begin{align}
  \mathbb{P}=\{\} \\
  \mathbb{C}=\{\} \\
  \mathbb{E}=\{\} \\
\end{align}
\begin{align}
  aaa
\end{align}
\begin{align}
  aaa
\end{align}


\section{記述できないパズルルール}\label{section:CannotDescribePuzzleRule}
今節では, \cref{chapter:Prepare}で定義したパズルルールでは記述できないパズルルールについて説明を行う.
現状, 既存のペンシルパズルのパズルルールを全て\cref{chapter:Prepare}で定義したパズルルールで記述することができるわけではない. いくつか例を挙げて補足を行う.
