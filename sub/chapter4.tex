\chapter{結言}
\cref{chapter:Definition}では, サイズ$m\times n$の平面グリッドを与えた際に, 種々の数学的定義を行うことでパズルルールが定義できることを主張した. また, パズルルールを構成する中枢を担う\textit{conditions}の具体的な構成方法について\cref{section:ConcreteConditions}で説明を行った. \cref{chapter:Demonstration}では本研究におけるパズルルールの定義を用いることで, 既存のパズルルールを数学的に記述することを可能にすることを示した. さらに, \cref{section:ConcreteConditions}で説明した\textit{conditions}の構成方法を用いることによって, 新規のパズルルールを作成できることを示した.

このパズルルールの定義を用いることで, 既存, 新規にかかわらずパズルルールの数学的記述ができる. さらに, \textit{conditions}の性質の分類, 抽象化を行うことでパズルルールの体系的な分類が期待される. これらが実現すればパズルルールの機械的な作成, あるいは自動作成が期待できる. 本論文中で触れた, \cref{theorem:AutoGeneration}の適用条件を明確にできれば, 自動作成したパズルルールの具体的な問題を作成できるため, 商業用ゲームへの移植が実現できる.

今後の展望として, まず本文中で触れたように\cref{theorem:AutoGeneration}の適用範囲について明確にすることが挙げられる. これに成功すれば, その適用範囲を満たしている\textit{codomain}と\textit{conditions}を満たす完成盤面を一つ考えることができれば, それは問題の存在と同値になり, より実用的になることが期待される.

さらに, ペンシルパズルの体系的な難易度評価の研究が挙げられる. 研究段階では\textit{codomain}とHIの選び方によりパズルルールの難易度が変わることが定性的に分かっている. Komiya-Kotani-Korekawa\cite{Komiya2010}はペンシルパズルの難易度を"Solution Path"という概念を考案し, 体系的に難易度評価を行っている. しかしこれはあるパズルルールが存在したときの問題の難易度であり, パズルルール自体の難易度を測る指標ではない. パズルルール自体の難易度を\textit{codomain}とHIから定量的に算出する方法があれば, 自動作成されたパズルルールがより実用的になることが期待される.

他に, ペンシルパズルの定義を本論文では平面グリッドとしたが, 実際には六角形グリッドであったり, 球体上にグリッドを配置したパズルも考えることができる. 研究段階で, 平面グリッドであるための必要十分条件が「盤面$B$において, あるグラフ$G_z$が存在したとき, $G_z \ni \forall p(i,j)$, $\text{cross}(p(i,j))\le3$」であると予想されている. この制約を外すことによって, 立体形状にグリッドを配置したパズルを体系的に考えることができると期待される.