\chapter*{{\rm \bf Abstract}}\label{chapter:Abstract}
パズルは, 世界中で多くの人々を魅了する娯楽の一つである. その形態は多岐にわたり, 様々なパズルが研究対象となっている. パズルの一つであるペンシルパズルは, 世界で最も人気のパズルジャンルのうちの一つである. ペンシルパズルは完全情報確定パズルと言われるものの一種であり, その計算量や解法アルゴリズム, あるいは問題の自動作成等については深く研究が進んでいる. しかし一つ抽象度を上げた, ペンシルパズルのパズルルールを自動的に作成したという先行研究は現時点で存在しない. 先行研究で確認されているように, 商業用ゲームでのマンネリ化を解消するためにはパズルが有効である. しかし, 一種類のパズルのみではマンネリ化を解消することは難しく, 生産コストの点から多種多様なパズルルールを生み出すことは難しい. そこで, パズルルールを自動作成することが求められている.

本研究はパズルルールの自動作成における課題解決のための研究である. パズルルールの自動作成における課題としてペンシルパズルのパズルルールに関し, 先行研究にて統一的な数学的記述の整理がなされていないことが挙げられる. そこで, 本研究の目的はパズルルールを数学的に定義し, それらにまつわる概念を整理することとした.

そこで本論文ではパズルルールにまつわる基礎的な概念として, 平面グリッドにおいて位置, 状態, 盤面などと数学的に定義した上で, パズルルールを記述するために四つの概念を導入した(\textit{codomain}, \textit{conditions}, \textit{Hidden Information}(HI), \textit{identification}). そして, それらの概念の組み合わせでパズルルールが数学的に定義できることを主張した. また, それらの数学的記述の妥当性を検証するために\cref{chapter:Demonstration}ではそれらを用いて既存のパズルルールを数学的に記述できることを実証した. さらに本研究の定義を用いて新しいパズルルールを作成し, そのパズルルールの問題が実際に存在することも示した.

本研究でのパズルルールの数学的な記述を用いることにより, 既存のパズルルールがどのような性質を持っているかの体系的な整理が期待される. また, その性質を分類, 抽象化することにより, それらの性質を持った新規のパズルルールを機械的に作成することが期待される. また, 先行研究の定理を本研究に適用できる制約条件を明確にすることで, 将来的に自動作成されたパズルルールから具体的なパズルの問題を自動作成も期待される. さらに, ペンシルパズルのパズルルール自体の難易度評価がなされた先行研究は存在しない. そこで, 本研究のパズルルールの数学的な記述を行うことにより, パズルルール自体の難易度評価の可能性についても説明した.