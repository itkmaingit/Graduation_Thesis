\documentclass[12pt, pdftex]{amsart}
\usepackage[margin=1in]{geometry}
\usepackage[utf8]{inputenc}
\usepackage{tikz}
\usetikzlibrary{arrows.meta,decorations.pathreplacing,decorations.markings,shapes,calc}
\usetikzlibrary{matrix,arrows,decorations.pathmorphing,backgrounds,decorations.markings,positioning}
% this package is used to rotate \sim.
\usepackage{graphics}
\newcommand{\rotatesim}{\rotatebox[origin=c]{90}{$\sim$}}
\usepackage[all]{xy}
\usepackage{comment}
\usepackage{amsmath}
% \usepackage{amscd}
\usepackage{color}

\usepackage{amssymb}
\usepackage{amsthm}
\usepackage{here}
\usepackage{amscd} 
\usepackage{tikz-cd}
\usepackage{mathrsfs}
\usepackage{mathtools}
\usepackage{mathabx}
\usepackage{scalefnt}
\usepackage{url}
\theoremstyle{plain}
\newtheorem{thm}{Theorem}[section]
\newtheorem{lem}[thm]{Lemma}
\newtheorem{cor}[thm]{Corollary}
\newtheorem{prop}[thm]{Proposition}
\newtheorem{conj}[thm]{Conjecture}
\theoremstyle{definition}
\newtheorem{ass}[thm]{Assumption}
\newtheorem{rem}[thm]{Remark}
\newtheorem{claim}[thm]{Claim}
\newtheorem{defn}[thm]{Definition}
\newtheorem{prob}[thm]{Problem}
\newtheorem{que}[thm]{Question}
\renewcommand{\thedefn}{}
\newtheorem{ex}[thm]{Example}
\renewcommand{\theex}{}
\renewcommand{\proofname}{\textbf{Proof}}
\numberwithin{equation}{section}
\def\A{{\mathbb A}}
\def\F{{\mathbb F}}
\def\Q{{\mathbb Q}}
\def\R{{\mathbb R}}
\def\Z{{\mathbb Z}}
\def\C{{\mathbb C}}
\def\P{{\mathbb P}}
\def\B{{\mathbb B}}
\def\S{{\mathfrak S}}
\def\id{\mathop{\mathrm{id}}\nolimits}
\def\End{\mathop{\mathrm{End}}\nolimits}
\def\Frob{\mathop{\mathrm{Frob}}\nolimits}
\def\Frac{\mathop{\mathrm{Frac}}\nolimits}
\def\Aut{\mathop{\mathrm{Aut}}\nolimits}
\def\Gal{\mathop{\mathrm{Gal}}\nolimits}
\def\Lie{\mathop{\mathrm{Lie}}\nolimits}
\def\Hom{\mathop{\mathrm{Hom}}\nolimits}
\def\Image{\mathop{\mathrm{Im}}\nolimits}
\def\Ker{\mathop{\mathrm{Ker}}\nolimits}
\def\Pic{\mathop{\mathrm{Pic}}\nolimits}
\def\NS{\mathop{\mathrm{NS}}\nolimits}
\def\GL{\mathop{\mathrm{GL}}\nolimits}
\def\SL{\mathop{\mathrm{SL}}\nolimits}
\def\SU{\mathop{\mathrm{SU}}\nolimits}
\def\Her{\mathop{\mathrm{Her}}\nolimits}
\def\Spec{\mathop{\mathrm{Spec}}\nolimits}
\def\rec{\mathop{\mathrm{rec}}\nolimits}
\def\Tr{\mathop{\mathrm{Tr}}\nolimits}
\def\Norm{\mathop{\mathrm{Norm}}\nolimits}
\def\det{\mathop{\mathrm{det}}\nolimits}
\def\dim{\mathop{\mathrm{dim}}\nolimits}
\def\div{\mathop{\mathrm{div}}\nolimits}
\def\Div{\mathop{\mathrm{Div}}\nolimits}
\def\vol{\mathop{\mathrm{vol}}\nolimits}
\def\Ind{\mathop{\mathrm{Ind}}\nolimits}
\def\Stab{\mathop{\mathrm{Stab}}\nolimits}
\def\rank{\mathop{\mathrm{rank}}\nolimits}
\def\Hom{\mathop{\mathrm{Hom}}\nolimits}
\def\diag{\mathop{\mathrm{diag}}\nolimits}
\def\antidiag{\mathop{\mathrm{antidiag}}\nolimits}
\def\chara{\mathop{\mathrm{char}}}
\def\resp{\mathop{\mathrm{resp}}}
\def\ss{\mathop{\mathrm{ss}}}

\def\ord{\mathrm{ord}}
\def\GIT{\mathrm{GIT}}
\def\K{\mathrm{K}}
\def\tor{\mathrm{tor}}
\def\BB{\mathrm{BB}}
\def\lev{\mathrm{lev}}



\def\L{\mathscr{L}}
\def\H{\mathscr{H}}
\def\g{\mathfrak{g}}
\def\k{\mathfrak{k}}
\def\N{\mathscr{N}}
\def\M{\mathcal{M}}
\def\F{\mathscr{F}}
\def\W{\mathscr{W}}
\def\OO{\mathscr{O}}
\def\V{\mathscr{V}}
\def\A{\mathcal{A}}
\def\D{\mathscr{D}}
\def\p{\mathfrak{p}}
\def\f{\mathfrak{f}}
\def\PP{\mathcal{P}}
\def\QQ{\mathcal{Q}}
\def\RR{\mathcal{R}}
\def\a{\alpha}
\def\b{\beta}
\def\g{\gamma}
\def\l{\langle}
\def\r{\rangle}
\def\Res{\mathop{\mathrm{Res}}\nolimits}
\def\GSpin{\mathop{\mathrm{GSpin}}\nolimits}
\def\CH{\mathop{\mathrm{CH}}\nolimits}
\def\Spin{\mathop{\mathrm{Spin}}\nolimits}
\def\exp{\mathop{\mathrm{exp}}\nolimits}
\def\Sp{\mathop{\mathrm{Sp}}\nolimits}
\def\SO{\mathop{\mathrm{SO}}\nolimits}
\def\SU{\mathop{\mathrm{SU}}\nolimits}
\def\Mp{\mathop{\mathrm{Mp}}\nolimits}
\def\Supp{\mathop{\mathrm{Supp}}\nolimits}
\def\Sym{\mathop{\mathrm{Sym}}\nolimits}
\def\PGL{\mathop{\mathrm{PGL}}\nolimits}
\def\U{\mathrm{U}}
\def\O{\mathrm{O}}
\def\v{\mathrm{vol}_{HM}}

\allowdisplaybreaks[4]

\newcommand{\transp}[1]{{}^{t}\!{#1}}
\newcommand{\defeq}{\vcentcolon=}
\renewcommand{\labelenumi}{(\arabic{enumi})}

\begin{document}
\title[Hyperbolicity of the moduli spaces of polarized Abelian surfaces]
{Provisional note: Hyperbolicity of the moduli spaces of polarized Abelian surfaces}
\author{Yota Maeda}
\email{y.maeda@math.kyoto-u.ac.jp}
\maketitle

\section{Introduction}
In this note, we shall study certain hyperbolicity of $\A^{\lev}_{1,t}=\H_2/\Gamma_{1,t}$.
\begin{prob}
\label{prob:main}
All rational or elliptic curves in $\A^{\lev}_{1,t}$ are contained in the Humbert surfaces $H_1$ or $H_2$ if $t$ is sufficiently large.
\end{prob}
We work on Problem \ref{prob:main} by extending the result \cite{Rou13} and constructing certain cusp forms by the Gritsenko lift.

We call a variety $X$ \textit{hyperbolic} if it is Kobayashi hyperbolic, i.e., the Kobayashi pseudo-metric is non-degenerate at any point $x\in X$.
For a subvariety $W\subset X$, we call $X$ hyperbolic modulo $W$ if $X\setminus W$ is hyperbolic.
Note that if $X$ is hyperbolic, then it is Brody hyperbolic, i.e., there is no non-constant map $f\colon:\C\to X$ and this implies that $X$ contains no  either rational or elliptic curves.
If $X$ is projective, then the converse is true (but we won't use this in this note).
Hence, Problem \ref{prob:main} is deduced to the following.
\begin{prob}
\label{prob:hyperbolicity}
A toroidal compactification $\overline{\A^{\lev}_{1,t}}^{\tor}$ is hyperbolic modulo its boundary $\overline{\A^{\lev}_{1,t}}^{\tor}\setminus \A^{\lev}_{1,t}$ and the Humbert surfaces $H_1$ and $H_2$.
\end{prob}
To solve this problem, first, we study the relationship between pseudo-metrics and cusp forms.
A pseudo-metric $\rho$ on $X$ is called \textit{distance decreasing} if for any map $f\colon (\Delta, g_p) \to (X, \rho)$, we have $f^{\star}\rho \leq g_p$.
Here, $(\Delta, g_p)$ is the unit disc equipped with the Poincar\'{e} metric.
Note that the Kobayashi metric is characterized the largest pseudo-metric among distance decreasing pseudo-metrics.

\section{General theorem}
Below, let $X\defeq\D/\Gamma$ be a quotient of Hermitian symmetric domain $\D$ by an arithmetic subgroup $\Gamma\subset\Aut(\D)$.
Here, we do \textit{not} assume that $\Gamma$ is neat.
We take a toroidal compactification $\overline{X}^{\tor}$ of $X$ and denote its boundary $D$.
Let $B=\cup_i B_i$ be the union of branch divisors of the map $\D\to\D/\Gamma$ and $n_i$ be the branch index of $B_i$.
\begin{thm}
\label{thm:hyperbolicity}
Let $x\in X\setminus B$.
We assume that there is a positive integer $\ell$ and a section $s\in H^0(\overline{X}^{\tor}, \ell(K_{\overline{X}^{\tor}}+D+\sum_i \frac{n_i-1}{n_i}B_i))$ satisfying 
\begin{enumerate}
\item $s(x)\neq 0$
\item $s$ vanishes on $D$ with multiplicity $>\frac{\ell}{\gamma}$
\item $s$ vanishes on $B$ with multiplicity $>0$.
\end{enumerate}
Then, there exists a pseudo-metric on $\overline{X}^{\tor}$ which is distance decreasing and  non-degenerate at $x\in X$.
In particular, the Kobayashi pseudo-metric on $\overline{X}^{\tor}$ is non-degenerate at $x\in X$.
\end{thm}
For the definition of the rational number $\gamma$, see \cite[p3]{Rou13}, but in our Siegel threefolds case, we can take $\gamma=\frac{1}{3}$.
\begin{proof}
We give a sketch of proof here.
Basically,  we follow the strategy of \cite[Proposition 2.2, 2.3, 2.4]{Rou13}.
We also take the same $\psi$ as in \cite{Rou13} and have to check that
\begin{enumerate}
\item $\psi$ vanishes on $D$ and $B$
\item $\psi\cdot g$ vanishes on $D$ and $B$.
\end{enumerate}
Here, $g$ is the Bergman metric on $\D$.
At the boundary, by taking neat cover, we can choose new coordinates $w_1,\cdots,  w_k$ where $w_i=z_i^{m_i}$ for some positive integer $m_i$, satisfying that the (irreducible components of) boundary is represented by the equation $(w_1\cdots w_k=0)$.
Then, the same discussion of \cite[Proposition 2.2, 2.3]{Rou13} holds, i.e., $\psi$ and $\psi\cdot g$ vanish on $D$.
At the branch divisors, the explicit form of the Bergman metric $(-4\frac{\partial^2}{\partial z_i\partial \overline{z_j}}\log(1-z_i\overline{z_j}))_{i,j}$, it follows easily that $\psi$ and $\psi\cdot g$ vanish on $B$ because $s$ does vanish on $B$.
Then, the existence of a distance decreasing pseudo-metric \cite[Proposition 2.4]{Rou13} also holds by the same discussion, hence this completes the proof.
\end{proof}

Now, let $D=\cup_i D_i$ be the irreducible decomposition of the toroidal boundary.
A little modification of Theorem \ref{thm:hyperbolicity} implies the following corollary.

\begin{cor}
\label{cor:hyperbolicity}
Let $x\in X\setminus B$.
We assume that there is a positive integer $\ell$ and a family of sections $s_i\in H^0(\overline{X}^{\tor}, \ell(K_{\overline{X}^{\tor}}+D_i+\sum_i \frac{n_i-1}{n_i}B_i))$ satisfying 
\begin{enumerate}
\item $s_i(x)\neq 0$
\item $s_i$ vanishes on $D_i$ with multiplicity $>\frac{\ell}{\gamma}$
\item $s_i$ vanishes on $B$ with multiplicity $>0$.
\end{enumerate}
Then, there exists a pseudo-metric on $\overline{X}^{\tor}$ which is distance decreasing and  non-degenerate at $x\in X$.
In particular, the Kobayashi pseudo-metric on $\overline{X}^{\tor}$ is non-degenerate at $x\in X$.
\end{cor}
\begin{proof}
Taking a partition of unity $\sum_i\a_i =1$ around the boundaries, it suffices to define $\psi_i\defeq ||s_i||_h^{2(\gamma-\epsilon)/\ell}$ and $\widetilde{g_i}\defeq \beta \psi_i g$, and use $\widetilde{g}\defeq \sum_i\a_i\widetilde{g_i}$ in \cite[Proposition 2.2, 2.3, 2.4]{Rou13}.
Here, we take $\beta \defeq \sup_i{\beta_i}$.
\end{proof}

\section{Cusp forms}
Below, we specify these to our situation.
By \cite{Got61a, Got61b, Uen71}, the branch divisors on $\A_{1,t}$ (it may need to take $t=p>2$) are only $H_1$ and $H_2$ with index 2; see also \cite[section 5A]{HKW93}.
Now, we have
\[K_{\overline{\A^{\lev}_{1,t}}^{\tor}}=3L-\frac{1}{2}H-D\]
in $\Pic(\overline{\A^{\lev}_{1,t}}^{\tor})\otimes\Q$ where $L$ is the automorphic line bundle of weight 1 (Hodge bundle) and $H\defeq H_1+H_2$.
In this note, \textit{weight} always means arithmetic weight.

Hence, by Theorem \ref{thm:hyperbolicity}, Problem \ref{prob:hyperbolicity} is deduced to the following problem, claiming the existence of certain cusp forms.
\begin{prob}
\label{prob:cusp_form}
For any $x\in \overline{\A^{\lev}_{1,t}}^{\tor}\setminus (D\cup H)$, there exists a section $s\in H^0(\overline{\A^{\lev}_{1,t}}^{\tor}, 3\ell L-\frac{1}{2}H-3\ell D)$, not vanishing at $x$ for some $\ell$.
\end{prob}
Here, in the above situation, we shall prove that any section $s\in H^0(\overline{\A^{\lev}_{1,t}}^{\tor}, 3\ell L-\frac{1}{2}H-3\ell D)$ does not vanish at $x$ for sufficiently large $t=p^r$ with $r\geq 0$.
On the other hand, we obtain 
\begin{align*}
3\ell L-\frac{1}{2}H-3\ell D&=\ell K_{\overline{\A^{\lev}_{1,t}}^{\tor}}+\frac{\ell-1}{2}H-2\ell D.\\
\end{align*}
Here, $\frac{\ell-1}{2}H$ is effective for $\ell>1$ and $(\ell K_{\overline{\A^{\lev}_{1,t}}^{\tor}}-2\ell D)\vert_{\A^{\lev}_{1,t}}$ is big because $\A^{\lev}_{1,t}$ is now of general type by for $t=p^r>>0$ by \cite{Erd04, GS96, OG89, San97} or \cite[section 2]{HS94}.
Hence, the set of the base point of the line bundle $3\ell L-\frac{1}{2}H-3\ell D$ is contained in $D$.

By the above discussion, Problem \ref{prob:cusp_form} is deduced to the following.
\begin{prob}
There exists a non-zero cusp form $f$ vanishing on $H$ with respect to $\Gamma_{1,t}$, satisfying the corresponding section $s\defeq s_f$ on $\overline{\A^{\lev}_{1,t}}^{\tor}$ has slope $\leq 1$ for $t=p^r>>0$.
\end{prob}
Here, \textit{slope} of $s_f$ is the quantity (weight of $f$)/(vanishing order of $s_f$ at $D$ on $\overline{\A^{\lev}_{1,t}}^{\tor}$).

On the other hand, by the injectivity of the Gritsenko lift and the dimension formula, there exists a non-zero cusp form of level $\Gamma_{1,p^r}$ if $p\geq 37$ and $p\neq 41,47,59,71$.
From now on, we shall study 
\begin{enumerate}
\item (Reflective obstruction:) $f$ vanishes on $H$
\item (Cusp obstruction:) $s_f$ has slope $\leq 1$.
\end{enumerate}



Below, we show a strategy to prove (1) and (2).

To treat (1), we apply Hirzebuch's proportionality principle, Hirzebruch-Riemann-Roch theorem and Kodaira vanishing theorem to the line bundle $\ell L-H-D$ on $\overline{\A^{\lev}_{1,t}}^{\tor}$.
This strategy was used in \cite{HS94}.
This implies that the space $H^0(\overline{\A^{\lev}_{1,p}}^{\tor}, \ell L-D-H)$ has positive dimension for large $p$.
Thus the rest problem is to pullback a section in $H^0(\overline{\A^{\lev}_{1,p}}^{\tor}, \ell L-D-H)$ along (2) and construct a low slope section on $\overline{\A^{\lev}_{1,p^r}}^{\tor}$.

To work on (2), we take a finite ramified cover $\overline{\A^{\lev}_{1,p^r}}^{\tor}\to \overline{\A^{\lev}_{1,p}}^{\tor}$.
We want to prove that this map ramifies along the boundary with large branch index at least $p^{2(r-1)}$.
This can be done by comparing the center of the unipotent part of $\Gamma_{1,p^r}$ with $\Gamma_{1,p}$. 

\textcolor{red}{Note: there actually exists a boundary component that does not ramify under the finite covering $\Gamma_{1,p^r}\to\Gamma_{1,p}$.
We can prove that ``standard" boundaries always ramify, but this is not enough for our case because $\Gamma_{1,p^r}$ is \textit{not} a normal subgroup of $\Gamma_{1,p}$ (or $\Sp_2(\Z)$), thus the branch indexes are \textit{not} the same for some orbits ($\Gamma_{1,p}/\Gamma_{1,p^r}$ is not a group!) unlike the case of the moduli space of the principally polarized $g$-dimensional abelian varieties with canonical level structure $\overline{\A_g(n)}$ \cite{Rou13} ($\Sp_g(\Z)/\Gamma(n)$ is a group, so the branch indexes under this group action are as the same and it suffices to consider the standard boundaries $F_0,\dots,F_{g-1}$).}

\textcolor{blue}{A sketch of proof of $\dim H^0(\overline{\A_{1,p}^{\lev}}^{\tor} \ell(kL-m D-\frac{1}{2}H))>0$ for $\ell, k, m, p>>0$ with $k/m<1$ is as follows.
Note that 
\[\dim H^0 (kL-D)=p^5k^3 + O(k^2).\]
Let $D_0$ be the central component and $\{D_i\}_{i=1}^r$ be the peripheral components of a toroidal boundary.
\begin{enumerate}
\item First, we shall show that
\[\dim H^0(\overline{\A_{1,p}^{\lev}}^{\tor} \ell(kL-(k+1) D)=\dim H^0(\overline{\A_{1,p}^{\lev}}^{\tor} (3kL-3(k+1) D)>0.\]
Here, for simplicity, let $m=k+1, \ell =3$.
Considering the Fourier-Jacobi expansion of Siegel modular forms of genus 2, the cusp obstructions can be estimated by the dimension of Jacobi forms.
Near a peripheral component $D_i$, the obstruction is 
\[\bigoplus_{m=0}^{3(k+1)}J_{3k, m}\]
and dimension running all $D_i$ has order $p^2 k^3 +O(k^2)$ by \cite[Proposition 3.2]{HS94}.
Near the central boundary component $D_0$, by \cite[Proposition 3.7]{HS94}, the dimension of obstruction due to $D_0$ has order $p^3 k^3 + O(k^2)$.
Hence, the cusp obstruction has order $p^2 k^3 + O(k^2)$ and this implies that this obstruction is small enough for sufficiently large $p$.
\item Second, we prove 
\[\dim H^0(\overline{\A_{1,p}^{\lev}}^{\tor} kL-mD-\frac{1}{2}H)=\dim H^0(\overline{\A_{1,p}^{\lev}}^{\tor} 2kL-2m D - H)>0,\]
where $\ell =2$ for simplicity.
Now, we have an exact sequence
\[0\to\OO(k L-mD-H)\to\OO(k L-mD)\to\OO_H(k L-mD)\to 0.\]
Thus, we have to prove that $h^0(k L-mD)>>h^0((k L-mD)\vert_H)$, but left-hand side can be estimated by (1) and has order $(p^5-p^2)k^3 +O(k^2)$.
The right-hand side can be computed \cite[Corollary 4.7, Theorem 4.19]{HS94}, for sufficiently divisible $k$ (maybe a multiple of 12), it follows $h^0((k L-mD)\vert_H)=0$.
\end{enumerate}
Therefore, for large $p$ and sufficiently divisible $k, \ell$, we conclude 
\[\dim H^0(\overline{\A_{1,p}^{\lev}}^{\tor} \ell(kL-m D-\frac{1}{2}H))>0.\]}

\subsection{Reflective obstruction}
Here, we show an outline of how to estimate the dimension of the space $H^0(\overline{\A^{\lev}_{1,p}}^{\tor}, \ell L-D-H)$.
First, we consider the exact sequence 
\[0\to\OO(-H)\to\OO\to\OO_H\to 0.\]
By twisting the appropriate factor, we obtain
\[0\to\OO(\ell L-D-H)\to\OO(\ell L-D)\to\OO_H(\ell L-D)\to 0.\]
Here, we take a cohomological long exact sequence.
By easy consequence of Kodaira vanishing theorem, since $\overline{\A^{\lev}_{1,p}}^{\tor}$ has only quotient singularities, we have
\[H^1(\overline{\A^{\lev}_{1,p}}^{\tor}, \ell L-D-H)=0\]
for sufficiently large $\ell$ (I might think that we have to take $\ell'(\ell L-D-H)$ for large $\ell'$ because of quotient singularities, but this is no problem for our purpose).
Hence, to prove that $\dim H^0(\overline{\A^{\lev}_{1,p}}^{\tor}, \ell L-D-H) >0$, it suffices to show that 
\begin{align}
\label{ineq:cusp_forms}
\dim H^0(\overline{\A^{\lev}_{1,p}}^{\tor}, \ell L-D)>\dim H^0(\overline{\A^{\lev}_{1,p}}^{\tor}, (\ell L-D)\vert_H).
\end{align}
The right-hand side of (\ref{ineq:cusp_forms}) can be estimated as 
\[c\ell^3+O(\ell^2)\]
by Hirzebruch's proportionality principle, the Kodaira vanishing theorem and the Hirzebruch-Riemann-Roch theorem.
Here, $c$ is a positive constant, depending only on the volume of $\Gamma_{1,p}$.
On the other hand, the left-hand side of (\ref{ineq:cusp_forms}) can be estimated as 
\[c'\ell^2+O(\ell)\]
in the same way for some positive integer $c'$ depending on the number of components of $H$ (in the case of $\overline{\A^{\lev}_{1,p}}^{\tor}$, this number is 2, thus it can be easily controlled).
Therefore, for sufficiently large $\ell$, we can obtain the result which claims that the reflective obstruction is small enough for large $\ell$.

\subsection{Cusp obstraction}
By the above explanation, we have to modify the strategy to prove that the map $\overline{\A^{\lev}_{1,p^r}}^{\tor}\to \overline{\A^{\lev}_{1,p}}^{\tor}$ ramifies at \textit{any} boundary components.  
Below, we show its modification.
Our goal is to construct a family of modular forms $f_i$ vanishing on $H$ and $D_i$ with slope 1.
Here, $D=D(p^r)\cup \cup_i D_i$ and $D(p^r)$ (resp. $D_i$) corresponds to the blow up of the Baily-Borel boundary $X(p^r)$ (resp. $X(1)$).
Note that the number of $D_i$ is order $p^2r$.
\begin{align}
\dim S_k(\Gamma_{1,t})&=\dim H^0(\overline{\A^{\lev}_{1,t}}^{\tor}, kL-D)=t^5k^3+O(k^2)\\
\dim \dim H^0(\overline{\A^{\lev}_{1,t}}^{\tor}, kL-kD_i)&=t^5k^3+O(k^2)\\
\dim \dim H^0(\overline{\A^{\lev}_{1,t}}^{\tor}, kL-D)&=t^5k^3+O(k^2)
\end{align}

\section{Conjecture}
\textcolor{blue}{\begin{conj}
Let $G$ be a semi-simple algebraic group and $\D=G(\R)/K$ be the associated Hermitian symmetric domain of dimension $N$.
Suppose that $\{\Gamma_i\}_i$ be a family of infinitely many arithmetic subgroups in $\Gamma\defeq G(\Z)$ where 
\[d_i\defeq [\Gamma : \Gamma_i]\to\infty\ (i\to\infty).\]
Then, $\overline{X_i}^{\tor}\defeq \overline{\D/\Gamma_i}^{\tor}$ is hyperbolic modulo branch divisors and boundary divisors if $d_i>>0$.
\end{conj}
\begin{proof}
Here, we write an idea.
The space of modular forms have dimension $h^0(kL)=d_ik^N+O(k^{N-1})$.
Also, cusp and reflective obstruction can be estimated as above.
Here, we have to pay attention to the number of branch divisors and cusps, however this should be not a big problem.
In the orthogonal or unitary cases, the number of branch divisors can be estimated by a purely lattice-theoretic method as in the paper by Ma or the author.
Alternatively, as in \cite[Corollary 4.7, Theorem 4.19]{HS94}, this might be not an obstruction.
\end{proof}}

\begin{que}
\begin{enumerate}
\item For the case of $t=p^r$ or general a composite number?
\item We hope that we can prove that the moduli space of quasi-polarized $K3$ surfaces with polarization degree $2d$ are hyperbolic for sufficiently large $d$ based on the work of Gritsenko-Hulek-Sankaran in the same way (or more general settings).
\end{enumerate}
\end{que}

\begin{thebibliography}{99}

\bibitem{Erd04}
C. Erdenberger,
The Kodaira dimension of certain moduli spaces of abelian surfaces,
Math. Nachr. 274/275 (2004), 32–39.

\bibitem{HKW93}
K. Hulek, C. Kahn, S. H. Weintraub, 
Moduli spaces of abelian surfaces: compactification, degenerations, and theta functions.
De Gruyter Expositions in Mathematics, 12. Walter de Gruyter \& Co., Berlin, 1993.

\bibitem{Got61a}
E. Gottschling,
Über die Fixpunkte der Siegelschen Modulgruppe,
Math. Ann. 143 (1961), 111–149.

\bibitem{Got61b}
E. Gottschling,
Über die Fixpunktuntergruppen der Siegelschen Modulgruppe,
Math. Ann. 143 (1961), 399–430.

\bibitem{GS96}
V. A. Gritsenko, G. K. Sankaran, 
Moduli of abelian surfaces with a (1,p2) polarisation,
Izv. Ross. Akad. Nauk Ser. Mat. 60 (1996), no. 5, 19–26.

\bibitem{HS94}
K. Hulek, G. K. Sankaran,
The Kodaira dimension of certain moduli spaces of abelian surfaces,
Compositio Math. 90 (1994), no. 1, 1–35.

\bibitem{OG89}
K. G. O'Grady,
On the Kodaira dimension of moduli spaces of abelian surfaces,
Compositio Math. 72 (1989), no. 2, 121–163.

\bibitem{Rou13}
E. Rousseau,
Hyperbolicity, automorphic forms and Siegel modular varieties,
arXiv:1302.4723.

\bibitem{San97}
G. K. Sankaran, 
Moduli of polarised abelian surfaces,
Math. Nachr. 188 (1997), 321–340.

\bibitem{San22}
G. K. Sankaran,
A supersingular coincidence,
Ramanujan J. 59 (2022), no. 2, 609–613.

\bibitem{Uen71}
K. Ueno, 
On fibre spaces of normally polarized abelian varieties of dimension 2. I. Singular fibres of the first kind,
J. Fac. Sci. Univ. Tokyo Sect. IA Math. 18 (1971), 37–95.

\end{thebibliography}

\end{document}