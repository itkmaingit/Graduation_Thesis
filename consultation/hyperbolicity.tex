\documentclass[12pt, pdftex]{amsart}
\usepackage[margin=1in]{geometry}
\usepackage[utf8]{inputenc}
\usepackage{tikz}
\usetikzlibrary{arrows.meta,decorations.pathreplacing,decorations.markings,shapes,calc}
\usetikzlibrary{matrix,arrows,decorations.pathmorphing,backgrounds,decorations.markings,positioning}
% this package is used to rotate \sim.
\usepackage{graphics}
\newcommand{\rotatesim}{\rotatebox[origin=c]{90}{$\sim$}}
\usepackage[all]{xy}
\usepackage{comment}
\usepackage{amsmath}
% \usepackage{amscd}
\usepackage{amssymb}
\usepackage{amsthm}
\usepackage{here}
\usepackage{amscd} 
\usepackage{tikz-cd}
\usepackage{mathrsfs}
\usepackage{mathtools}
\usepackage{mathabx}
\usepackage{scalefnt}
\usepackage{url}
\theoremstyle{plain}
\newtheorem{thm}{Theorem}[section]
\newtheorem{lem}[thm]{Lemma}
\newtheorem{cor}[thm]{Corollary}
\newtheorem{prop}[thm]{Proposition}
\newtheorem{conj}[thm]{Conjecture}
\theoremstyle{definition}
\newtheorem{ass}[thm]{Assumption}
\newtheorem{rem}[thm]{Remark}
\newtheorem{claim}[thm]{Claim}
\newtheorem{defn}[thm]{Definition}
\newtheorem{prob}[thm]{Problem}
\newtheorem{que}[thm]{Question}
\renewcommand{\thedefn}{}
\newtheorem{ex}[thm]{Example}
\renewcommand{\theex}{}
\renewcommand{\proofname}{\textbf{Proof}}
\numberwithin{equation}{section}
\def\A{{\mathbb A}}
\def\F{{\mathbb F}}
\def\Q{{\mathbb Q}}
\def\R{{\mathbb R}}
\def\Z{{\mathbb Z}}
\def\C{{\mathbb C}}
\def\P{{\mathbb P}}
\def\B{{\mathbb B}}
\def\SS{{\mathfrak S}}
\def\id{\mathop{\mathrm{id}}\nolimits}
\def\End{\mathop{\mathrm{End}}\nolimits}
\def\Frob{\mathop{\mathrm{Frob}}\nolimits}
\def\Frac{\mathop{\mathrm{Frac}}\nolimits}
\def\Aut{\mathop{\mathrm{Aut}}\nolimits}
\def\Gal{\mathop{\mathrm{Gal}}\nolimits}
\def\Lie{\mathop{\mathrm{Lie}}\nolimits}
\def\Hom{\mathop{\mathrm{Hom}}\nolimits}
\def\Image{\mathop{\mathrm{Im}}\nolimits}
\def\Ker{\mathop{\mathrm{Ker}}\nolimits}
\def\Pic{\mathop{\mathrm{Pic}}\nolimits}
\def\NS{\mathop{\mathrm{NS}}\nolimits}
\def\GL{\mathop{\mathrm{GL}}\nolimits}
\def\SL{\mathop{\mathrm{SL}}\nolimits}
\def\SU{\mathop{\mathrm{SU}}\nolimits}
\def\Her{\mathop{\mathrm{Her}}\nolimits}
\def\Spec{\mathop{\mathrm{Spec}}\nolimits}
\def\rec{\mathop{\mathrm{rec}}\nolimits}
\def\Tr{\mathop{\mathrm{Tr}}\nolimits}
\def\Norm{\mathop{\mathrm{Norm}}\nolimits}
\def\det{\mathop{\mathrm{det}}\nolimits}
\def\dim{\mathop{\mathrm{dim}}\nolimits}
\def\div{\mathop{\mathrm{div}}\nolimits}
\def\Div{\mathop{\mathrm{Div}}\nolimits}
\def\vol{\mathop{\mathrm{vol}}\nolimits}
\def\Ind{\mathop{\mathrm{Ind}}\nolimits}
\def\Stab{\mathop{\mathrm{Stab}}\nolimits}
\def\rank{\mathop{\mathrm{rank}}\nolimits}
\def\Hom{\mathop{\mathrm{Hom}}\nolimits}
\def\diag{\mathop{\mathrm{diag}}\nolimits}
\def\antidiag{\mathop{\mathrm{antidiag}}\nolimits}
\def\chara{\mathop{\mathrm{char}}}
\def\resp{\mathop{\mathrm{resp}}}
\def\ss{\mathop{\mathrm{ss}}}

\def\ord{\mathrm{ord}}
\def\GIT{\mathrm{GIT}}
\def\K{\mathrm{K}}
\def\tor{\mathrm{tor}}
\def\BB{\mathrm{BB}}
\def\Exc{\mathrm{Exc}}
\def\lev{\mathrm{lev}}

\def\E{\mathscr{E}}

\def\L{\mathscr{L}}
\def\H{\mathscr{H}}
\def\g{\mathfrak{g}}
\def\k{\mathfrak{k}}
\def\N{\mathscr{N}}
\def\M{\mathcal{M}}
\def\F{\mathscr{F}}
\def\W{\mathscr{W}}
\def\OO{\mathscr{O}}
\def\V{\mathscr{V}}
\def\A{\mathcal{A}}
\def\D{\mathscr{D}}
\def\p{\mathfrak{p}}
\def\f{\mathfrak{f}}
\def\PP{\mathcal{P}}
\def\QQ{\mathcal{Q}}
\def\RR{\mathcal{R}}
\def\a{\alpha}
\def\b{\beta}
\def\g{\gamma}
\def\l{\langle}
\def\r{\rangle}
\def\Res{\mathop{\mathrm{Res}}\nolimits}
\def\GSpin{\mathop{\mathrm{GSpin}}\nolimits}
\def\CH{\mathop{\mathrm{CH}}\nolimits}
\def\Spin{\mathop{\mathrm{Spin}}\nolimits}
\def\exp{\mathop{\mathrm{exp}}\nolimits}
\def\Sp{\mathop{\mathrm{Sp}}\nolimits}
\def\SO{\mathop{\mathrm{SO}}\nolimits}
\def\SU{\mathop{\mathrm{SU}}\nolimits}
\def\Mp{\mathop{\mathrm{Mp}}\nolimits}
\def\Supp{\mathop{\mathrm{Supp}}\nolimits}
\def\Sym{\mathop{\mathrm{Sym}}\nolimits}
\def\PGL{\mathop{\mathrm{PGL}}\nolimits}
\def\PSL{\mathop{\mathrm{PSL}}\nolimits}
\def\U{\mathrm{U}}
\def\O{\mathrm{O}}
\def\v{\mathrm{vol}_{HM}}

\renewcommand{\contentsname}{Projects}

\allowdisplaybreaks[4]

\newcommand{\transp}[1]{{}^{t}\!{#1}}
\newcommand{\defeq}{\vcentcolon=}
\renewcommand{\labelenumi}{(\arabic{enumi})}



%%%%%%%%%%%%%%%%%%%%%%%%%%%%%%%%%%%%%%%%%%%%%%%%% FOOTNOTES %%%%%%%%%%%%%%%%%%%%%%%%%%%%%%%%%%%%%%%%%%%%%%%%%%

\newcommand{\marg}[1]{\normalsize{{\color{red}\footnote{#1}}}{\marginpar[\vskip -.3cm {\color{red}\hfill\tiny{\thefootnote}\normalsize$\implies$}]{\vskip -.3cm{ \color{red}$\impliedby$\tiny\thefootnote}}}}
\newcommand{\Gregory}[1]{\marg{\color{red}(Gregory): #1}}
\newcommand{\Yota}[1]{\marg{\color{blue}(Yota): #1}}
\newenvironment{aside}{\begin{quote}\sffamily}{\end{quote}}

\title[Hyperbolicity of the moduli spaces of $(1,p)$-polarized abelian surfaces]
{Hyperbolicity of the moduli spaces of $(1,p)$-polarized 
abelian surfaces}
\author{Yota Maeda}
\address{Y.M: Department of Mathematics, Faculty of Science, Kyoto University, Kyoto 606-8502, Japan/Advanced Research Laboratory, R\&D Center, Sony Group Corporations, 1-7-1 Konan, Minato-ku, Tokyo, 108-0075, Japan}
\email{y.maeda.math@gmail.com\\ y.maeda@math.kyoto-u.ac.jp}
\author{Gregory Sanakaran}
\address{G.S: }
\email{}



\begin{document}

\maketitle

\begin{abstract}
    We prove that the moduli spaces of $(1,p)$-polarized abelian surfaces with a canonical level structure are Kobayashi-hyperbolic modulo the Humbert surfaces if $p$ is sufficiently large.
    Moreover, we consider a conjecture claiming a hyperbolicity for a family of arithmetic subgroups.
    This is analogue to the conjecture by Brunebarbe.
\end{abstract}

\section{Introduction}
Hyperbolicity is an important notion to study the birational geometrical property of projective varieties and their subvarieties.
In this context, the Lang conjecture plays a critical role.
\begin{conj}[{\cite{GG79, La87}}]
Let $X$ be a projective variety over $\C$.
Then, $X$ is of general type if and only if  there exists a proper algebraic subset $\Exc(X)\subsetneq X$ such that
\begin{enumerate}
    \item there is no non-constant holomorphic map $\C\to X$ whose image is not contained in $\Exc(X)$, and
    \item every subvariety of $X$ of non-general type is contained in $\Exc(X)$. 
\end{enumerate}
    
\end{conj}
In this paper, for modular varieties, we will see that $\Exc(X)$ is closely related to the branch divisors of the uniformization maps and toroidal boundaries.
In particular, we work on $\A_{1,p}^{\lev}$, the moduli spaces of $(1,p)$-polarizaed abelian surfaces with a canonical level structure, studied in \cite{HS94, HKW93}.
Our main result in this paper is the following theorem.
\begin{thm}[{Theorem \ref{thm:hyperbolicity_abelian_surfac}}]
\label{mainthm:hyperbolicity_abelian_surface}
If $p$ is sufficiently large, then, any toroidal compactification $\overline{\A_{1,p}^{\lev}}$ is Kobayashi-hyperbolic modulo its toroidal boundary and the Humbert surfaces.
\end{thm}

\subsection*{Acknowledgements}
The first author is supported by JST ACT-X JPMJAX200P.

\section{Criterion of the Kobayashi-hyperbolic for modular varieties}
Below, let $X\defeq\D/\Gamma$ be a quotient of Hermitian symmetric domain $\D$ by an arithmetic subgroup $\Gamma\subset\Aut(\D)$.
In this paper, we do \textit{not} assume that $\Gamma$ is neat.
We take a toroidal compactification $\overline{X}$ of $X$ and denote its boundary $D$.
Let $B=\cup_i B_i$ be the union of branch divisors of the map $\D\to\D/\Gamma$ and $n_i$ be the branch index of $B_i$.
Rousseau \cite[Theorem 2.1]{Ro16} proved a criterion, claiming that modular varieties are Kobayashi-hyperbolic in terms of specific modular forms.
He assumed that $\Gamma$ is neat.
For our purpose, first, we develop this to non-neat cases.
We recall the setting of \cite{Ro16}; let $g$ be the Bergman metric on $\D$ satisfying the K\"ahler-Einstein property $\mathrm{Ric}(g)=-g$ and $\gamma$ be a positive rational number such that $g$ has holomorphic sectional curvature less than or equal $-\gamma$.
\begin{thm}
\label{thm:hyperbolicity}
Let $x\in X\setminus B$.
We assume that there is a positive integer $k$ and a section $s\in H^0\bigl(\overline{X}, k(K_{\overline{X}}+D+\sum_i \frac{n_i-1}{n_i}B_i)\bigr)$ satisfying 
\begin{enumerate}
\item $s(x)\neq 0$.
\item $s$ vanishes on $D$ with multiplicity $>\frac{k}{\gamma}$.
\item $s$ vanishes on $B$.
\end{enumerate}
Then, there exists a pseudo-metric on $\overline{X}$ which is distance decreasing and  non-degenerate at $x\in X$.
In particular, the Kobayashi pseudo-metric on $\overline{X}$ is non-degenerate at $x\in X$.
\end{thm}
\Yota{Below, generally we can comment as this.}
\Yota{We might have to consider the effects of singularities; should we assume the section $s$ vanishes on the singular locus, described in \cite[Section 4]{HS94}?
Cadorel, who was a student of Rousseau, also studied the non-neat version of Rousseau's results; for instance, 
\begin{itemize}
\item [CRT19a] B. Cadorel, E. Rousseau, B. Taji,
\textit{Hyperbolicity of singular spaces},
J. Éc. polytech. Math. 6 (2019), 1–18.
\item [CRT19b] B. Cadorel,  E. Rousseau,  B. Taji,
\textit{Hyperbolicity of singular spaces},
J. Éc. polytech. Math. 6 (2019), 1–18.
    \item [Ca22] B. Cadorel, \textit{Subvarieties of quotients of bounded symmetric domains}, 
Math. Ann. 384 (2022), no. 1-2, 419–456.
\end{itemize}
It may be necessary to impose conditions at the singularities or refer to his work by using desingularizations of the singularities; see also [CRT19a, Theorem A], [CRT19b, Theorem B] and [Ca22, Theorem 5, 6].}

\begin{proof}
Basically,  we follow the strategy of \cite[Proposition 2.2, 2.3, 2.4]{Ro16}.
We also take the same $\psi$ as in \cite[Section 2]{Ro16} and have to check that
\begin{enumerate}
\item $\psi$ vanishes on $D$ and $B$
\item $\psi\cdot g$ vanishes on $D$ and $B$.
\end{enumerate}
Along the boundary, by taking neat cover, we can choose new coordinates $w_1,\cdots,  w_k$ where $w_i=z_i^{m_i}$ for some positive integer $m_i$, satisfying that the (irreducible components of) boundary is represented by the equation $(w_1\cdots w_k=0)$ by \cite{Mu77}.
Then, the same discussion of \cite[Proposition 2.2, 2.3]{Ro16} holds, i.e., $\psi$ and $\psi\cdot g$ vanish on $D$.
Around the branch divisors, the explicit form of the Bergman metric $(-4\frac{\partial^2}{\partial z_i\partial \overline{z_j}}\log(1-z_i\overline{z_j}))_{i,j}$ implies quickly that $\psi$ and $\psi\cdot g$ vanish on $B$ because $s$ does vanish on $B$.
Then, the existence of a distance decreasing pseudo-metric \cite[Proposition 2.4]{Ro16} also holds by the same discussion, hence this completes the proof.
\end{proof}

Now, let $D=\cup_j D_j$ be the irreducible decomposition of the toroidal boundary.
A little modification of Theorem \ref{thm:hyperbolicity} implies the following corollary.

\begin{cor}
\label{cor:hyperbolicity_family}
Let $x\in X\setminus B$.
We assume that there is a positive integer $k$ and a family of sections $s_j\in H^0\bigl(\overline{X}, k(K_{\overline{X}}+D_j+\sum_i \frac{n_i-1}{n_i}B_i)\bigr)$ satisfying 
\begin{enumerate}
\item $s_j(x)\neq 0$.
\item $s_j$ vanishes on $D_j$ with multiplicity $>\frac{k}{\gamma}$.
\item $s_j$ vanishes on $B$.
\end{enumerate}
Then, there exists a pseudo-metric on $\overline{X}$ which is distance decreasing and  non-degenerate at $x\in X$.
In particular, the Kobayashi pseudo-metric on $\overline{X}$ is non-degenerate at $x\in X$.
\end{cor}
\begin{proof}
First, by taking an open covering $\{U_k^{(j)}\}_k$ of $D_j$, we cover $\overline{X}$ by $\{U_k^{(j)}\}_k$ and $\{X\}$.
Then, we can pick a partition of unity $\sum_{j,k} e_k^{(j)} + e_X =1$ satisfying $\Supp(e^{(k)}_j)\subset D_j$, $\Supp(e_X)\subset X$ and each function decrease as the exponential function outside their support.
Now, we define $\psi_j\defeq ||s_j||_h^{2(\gamma-\epsilon)/k}$ and $\widetilde{g_j}\defeq \beta \psi_j g$, and use $\widetilde{g}\defeq \sum_je^{(j)}_1\widetilde{g_j} + e_X\widetilde{g_1}$.
Here, we took $\beta \defeq \sup_i{\beta_i}$.
Then, as in the proof of Theorem \ref{thm:hyperbolicity}, these function satisfies \cite[Proposition 2.2, 2.3, 2.4]{Ro16}.
In particular, this concludes that $\widetilde{g}$ is distance decreasing and non-degenerate at $x$.
\end{proof}

Below, we consider moduli spaces of some abelian surfaces, and we can take $\gamma=1/3$ in the above theorem; see \cite[Section 4]{Ro16}.

\section{Cusp obstructions}
In this section, we shall estimate the obstruction by toroidal boundaries.
Let us denote $X\defeq\A_{1,p}^{\lev}$ in this and the next section and  recall the setting of \cite{HS94}.
We denote by $D(p)$ the \textit{central component}, a blow-up of $X(p)$, and $D_1,\cdots D_{(p^2-1)/2}$ the \textit{peripheral components}, blow-ups of $X(1)$; for the definitions, see \cite[Section 1]{HS94}.
Let $D'\defeq D_1\cup \dots \cup D_{(p^2-1)/2}$.
Note that each $D_i$ is transitively acted by a group $\Gamma_{1,p}^0/\Gamma_{1,p}$; introduced in \cite[Section 1]{HS94}, hence to estimate the obstructions, it suffices to consider one peripheral component $D_1$.

Below, we show that there are sufficiently many cusp forms of slope less than 1 if $p$ is large.
In detail, our goal in this section is:
\begin{prop}
\label{prop:cusp_obstruction}
    The cusp obstructions are estimated as 
    \[\dim H^0(\overline{X}, k(3\L-(m+3))D)\geq \frac{(m+3)^2}{144}\{(2m+15)(p^2-1) + 2\mu(9p+2(m+3))\}k^3+O(k^2).\]
    Here, $\mu$ is the order of $\PSL_2(\Z/p\Z)$.
\end{prop}
\begin{proof}
    This follows from Proposition \ref{prop:peripheral_obstruction} and \ref{prop:central_obstruction}.
\end{proof}

\begin{prop}
\label{prop:peripheral_obstruction}
For $m>0$, the following estimation holds:
\[\dim H^0\bigl(\overline{X}, k(3\L-(m+3))D'\bigr) \geq \frac{(p^2-1)}{144}(m+3)^2(2m+15)k^3 + O(k^2)\]
\end{prop}
\begin{proof}
    We follow the strategy of \cite[Proposition 3.3]{HS94}.
Let $J_{k,t}$ be the space of Jacobi forms of weight $k$ and index $t$.
By considering the Fourier-Jacobi expansion, the obstruction is estimated as 
\[\bigoplus_{t=0}^{k(m+3)-1}J_{k,t}.\]
For even $k$, we have
\begin{align*}
\sum_{t=0}^{k(m+3)-1}\dim J_{3k,t}&\leq \sum_{i=0}^{k(m+3)-1}\{k(m+3)-i\}\dim M_{3k+2i}\\
&\leq\frac{(m+3)^2}{72}(2m+15)k^3 + O(k^2).
\end{align*}
Now, there are $(p^2-1)/2$ peripheral components, thus we obtain the claim.
\end{proof}


\begin{prop}
\label{prop:central_obstruction}
For $m>0$, the following estimation holds:
\[\dim H^0\bigl(\overline{X}, k(3\L-(m+3))D(p)\bigr) \geq \frac{\mu(m+3)^2}{24}\{3p+\frac{2}{3}(m+3)\}k^3 + O(k^2)\]
\end{prop}
\begin{proof}
This can be followed by a similar discussion of \cite[Proposition 3.7]{HS94}, combined with \cite[Proposition 3.6]{HS94}.
\end{proof}


\begin{rem}
We note that if we take $m=0$, Proposition \ref{prop:peripheral_obstruction} (resp. Proposition \ref{prop:central_obstruction}) coincides with \cite[Proposition 3.3]{HS94} (resp. \cite[Proposirion 3.7]{HS94}).
\end{rem}

\section{Reflective obstructions}
Let $H\defeq H_1\cup H_2$ be the Humbert surface.
From \cite{Go61a, Go61b, Ue71}, the branch divisors of the map $\mathbb{H}_2\to\A_{1,p}^{\lev}$ is $H$ with index 2; see also \cite[Section 2]{HKW93}.
There is an exact sequence 
\[0\to \OO\bigl(\frac{k}{2}(6\L - 2(m+3)D - H)\bigr)\to \OO\bigl(\frac{k}{2}(6\L - 2(m+3)D)\bigr)\to \OO_H\bigl(\frac{k}{2}(6\L - 2(m+3)D)\bigr)\to 0.\]

\begin{prop}
\label{prop:reflective_obstruction}
If $p\geq 5$ and $k$ is sufficiently divisible, then
    \[h^0\bigl(\frac{k}{2}(6\L - 2(m+3)D)\vert_H\bigr)=0.\]
\end{prop}
\begin{proof}
    This follows from \cite[Theorem 4.19, 4.25]{HS94}.
    \Yota{Hulek-Sankaran used the desingularization, hence we have to recheck this.}
\end{proof}

\section{Proof of the main theorem}
Before discussing our main result, we shall prepare some Lemma.

\begin{lem}
\label{lem:base_point}
    We assume that $\dim H^0(\overline{\A_{1,p}^{\lev}}, k(3\L-(m+3)D-\frac{1}{2}H))\neq 0$.
    Then, if $p$ is sufficiently large, for any point $x\in \overline{\A_{1,p}^{\lev}}\setminus (D\cup H)$, there exists a section $s\in H^0(\overline{\A_{1,p}^{\lev}}, k(3\L-(m+3)D-\frac{1}{2}H))$, not vanishing at $x$ for sufficiently large $k$.
\end{lem}
\begin{proof}
By the standard application of Hirzebruch's proportionality principle, we have
\[K_{\overline{\A^{\lev}_{1,p}}} = 3\L -\frac{1}{2}H-D.\]
Thus, it follows 
\begin{align*}
3k L-k(m+3)D-\frac{k}{2}H&=k K_{\overline{\A^{\lev}_{1,p}}}-k(m+2)D.
\end{align*}
Now, $(kK_{\overline{\A^{\lev}_{1,p}}}-k(m+2)D)\vert_{\A^{\lev}_{1,p}}$ is big because $\A^{\lev}_{1,p}$ is now of general type if $p$ is large by \cite{HS94}.
Hence, the set of the base point of the line bundle $3k L-k(m+3)D-\frac{k}{2}H$ is contained in $D$.
\end{proof}

\begin{thm}
\label{thm:hyperbolicity_abelian_surfac}
If $p$ is sufficiently large, then $\overline{\A_{1,p}^{\lev}}$ is Kobayashi-hyperbolic modulo its boundary and the Humbert surfaces.
\end{thm}
\begin{proof}
    By Proposition \ref{prop:cusp_obstruction} and \ref{prop:reflective_obstruction}, we obtain
    \[\dim H^0 (\overline{\A_{1,p}^{\lev}}, k(3\L-(m+3)D-\frac{1}{2}H)) \geq \frac{\mu(m+3)^2}{24}\{3p+\frac{2}{3}(m+3)\}k^3 + O(k^2)\]
    if $p\geq 5$ and $k$ is sufficiently divisible.
    Hence, combined with Lemma \ref{lem:base_point}, Theorem \ref{thm:hyperbolicity} concludes the proof.
\end{proof}

\section{General conjecture}
\begin{conj}
Let $G$ be a semi-simple algebraic group and $\D=G(\R)/K$ be the associated Hermitian symmetric domain of dimension $N$.
Suppose that $\{\Gamma_i\}_{i=1}^{\infty}$ be a family of infinitely many arithmetic subgroups in $\Gamma\defeq G(\Z)$ where 
\[d_i\defeq [\Gamma : \Gamma_i]\to\infty\ (i\to\infty).\]
Then, $\overline{X_i}\defeq \overline{\D/\Gamma_i}$ is hyperbolic modulo branch divisors and boundary divisors if $d_i>>0$.
\end{conj}
\begin{proof}
Here, we write an idea.
The space of modular forms has dimension $h^0(k\L)=d_ik^N+O(k^{N-1})$.
Also, cusp and reflective obstruction can be estimated as above.
Here, we have to pay attention to the number of branch divisors and cusps, however this should be not a big problem.
In the orthogonal or unitary cases, the number of branch divisors can be estimated by a purely lattice-theoretic method as in the paper by Ma or the author.
Alternatively, as in \cite[Corollary 4.7, Theorem 4.19]{HS94}, this might be not an obstruction.
\end{proof}

\begin{thebibliography}{99}

\bibitem[Bu20]{Bu20}
Y. Brunebarbe, 
\textit{A strong hyperbolicity property of locally symmetric varieties},
Ann. Sci. Éc. Norm. Supér. (4) 53 (2020), no. 6, 1545–1560.

\bibitem[Er04]{Er04}
C. Erdenberger,
``\textit{The Kodaira dimension of certain moduli spaces of abelian surfaces}",
Math. Nachr. 274/275 (2004), 32–39.

\bibitem[HKW93]{HKW93}
K. Hulek, C. Kahn, S. H. Weintraub, 
``\textit{Moduli spaces of abelian surfaces: compactification, degenerations, and theta functions}",
De Gruyter Expositions in Mathematics, 12. Walter de Gruyter \& Co., Berlin, 1993.

\bibitem[Go61a]{Go61a}
E. Gottschling,
``\textit{Über die Fixpunkte der Siegelschen Modulgruppe}",
Math. Ann. 143 (1961), 111–149.

\bibitem[Go61b]{Go61b}
E. Gottschling,
``\textit{Über die Fixpunktuntergruppen der Siegelschen Modulgruppe}",
Math. Ann. 143 (1961), 399–430.

\bibitem[GG79]{GG79}
M. Green,  P. Griffiths, 
\textit{Two applications of algebraic geometry to entire holomorphic mappings},
The Chern Symposium 1979 (Proc. Internat. Sympos., Berkeley, Calif., 1979), pp. 41–74, Springer, New York-Berlin, 1980.

\bibitem[GS96]{GS96}
V. A. Gritsenko, G. K. Sankaran, 
``\textit{Moduli of abelian surfaces with a $(1,p^2)$ polarisation},
Izv. Ross. Akad. Nauk Ser. Mat. 60 (1996), no. 5, 19–26.

\bibitem[HS94]{HS94}
K. Hulek, G. K. Sankaran,
``\textit{The Kodaira dimension of certain moduli spaces of abelian surfaces}",
Compositio Math. 90 (1994), no. 1, 1–35.

\bibitem[La87]{La87}
S. Lang,
\textit{Introduction to complex hyperbolic spaces},
Springer-Verlag, New York, 1987.

\bibitem[Mu77]{Mu77}
D. Mumford,
\textit{Hirzebruch's proportionality theorem in the noncompact case},
Invent. Math. 42 (1977), 239–272.

\bibitem[OG89]{OG89}
K. G. O'Grady,
``\textit{On the Kodaira dimension of moduli spaces of abelian surfaces}",
Compositio Math. 72 (1989), no. 2, 121–163.

\bibitem[Ro16]{Ro16}
E. Rousseau,
``\textit{Hyperbolicity, automorphic forms and Siegel modular varieties}",
Ann. Sci. Éc. Norm. Supér. (4) 49 (2016), no. 1, 249–255.

\bibitem[Sa97]{Sa97}
G. K. Sankaran, 
``\textit{Moduli of polarised abelian surfaces}",
Math. Nachr. 188 (1997), 321–340.

\bibitem[Sa22]{Sa22}
G. K. Sankaran,
``\textit{A supersingular coincidence}",
Ramanujan J. 59 (2022), no. 2, 609–613.

\bibitem[Ue71]{Ue71}
K. Ueno, 
``\textit{On fibre spaces of normally polarized abelian varieties of dimension 2. I. Singular fibres of the first kind}",
J. Fac. Sci. Univ. Tokyo Sect. IA Math. 18 (1971), 37–95.

\end{thebibliography}

\end{document}